\documentclass[12pt, a4paper]{article}

\usepackage{graphicx} % Required for inserting images

\usepackage[ngerman]{babel}

%Vormatierung 
\usepackage[margin=1in]{geometry}
\setlength{\parskip}{1em}
\setlength{\parindent}{0em}
\usepackage{ragged2e}
%header
\usepackage{fancyhdr}
\pagestyle{fancy}
\fancyhead{} % erst alles löschen
\fancyhead[L]{Tätigkeitsanalyse} % links
\fancyhead[R]{\thepage}
\renewcommand{\headrulewidth}{0.25pt}
% footer
\fancyfoot{} 
%Tabellen Abstand 
\setlength{\textfloatsep}{2\textfloatsep}
\setlength{\intextsep}{2\intextsep}
%tabellen . Ausrichtung mit S Spalte
\usepackage{siunitx}
\sisetup{%
  table-format = 3.2,   % 3 vor und 2 nach 
  detect-all          
}

%Zitate 
\usepackage[style=german]{csquotes}
\usepackage[style=apa]{biblatex}
\usepackage{hyperref}
\addbibresource{AO.bib}

%Tabellen Setup
\usepackage{booktabs}
\usepackage{threeparttable}
\usepackage{tabularx}
\newcolumntype{Y}{>{\centering\arraybackslash}X}
\newcolumntype{Z}{>{\raggedleft\arraybackslash}X}

\usepackage{caption}
\DeclareCaptionLabelSeparator{newlinewithvspace}{\\[2ex]}
\captionsetup{
  format=plain,
  labelfont=bf,
  textfont=it,
  labelsep=newlinewithvspace,
  singlelinecheck=false,  % wichtig, damit Zeilenumbruch auch bei kurzen Titeln funktioniert
  justification=raggedright
}

\usepackage{lipsum}

%Liste 
\usepackage[most]{tcolorbox}
\usepackage{xcolor}
\usepackage{enumitem}
\usepackage{fontawesome5}

%\setlist{nosep, left=0pt}
%\setlist[itemize,1]{label=\faChevronRight}
%\setlist[itemize,2]{label=\faAngleRight}


\begin{document}

\begin{titlepage}
    \begin{center}
        %\vspace*{0.25in}
        \large SS20XX 2000XX-X Arbeits- und Organisationspsychologie \\
        \vspace{1in}
        \Large Tätigkeitsanalyse \\
        \LARGE Anforderungen und Ressourcen:
        Handlungsempfehlungen zur Verbesserung der
        Arbeitssituation \\
        \vspace*{1in}
        \large 
        Vorname Name, Matrikelnummer \\
        Vorname Name, Matrikelnummer \\
        Vorname Name, Matrikelnummer \\
        Vorname Name, Matrikelnummer \\
        \vfill
        Lehrveranstaltungsleitung: \\
        Univ.-Prof. Dr. Christian Korunka \\
        Arabella Mühl, MSc \\
        \vspace{1in}   
    \end{center}
\end{titlepage}

\tableofcontents

\newpage

%\RaggedRight
\section{Einleitung}

\subsection{Beschreibung der Tätigkeit}


Die untersuchte Person, Frau Z., ist seit dem Jahr 2022 als Flugbegleiterin bei der Austrian Airlines, 
einem international tätigen Luftfahrtunternehmen mit Sitz in Niederösterreich, 
tätig. Das Unternehmen bietet Linien und Charterflüge im europäischen und interkontinentalen Raum. 
Es verfügt über ein großes Streckennetz und legt den Fokus auf Sicherheit und Servicequalität. 
Der Hauptsitz befindet sich nahe Wien, jedoch gibt es mehrere Unternehmensstandorte in Europa, an denen Flugbegleiter*innen zum Einsatz kommen können.

Ein zentrales Merkmal der Organisationsstruktur ist die hierarchische Gliederung im Bordbetrieb.
Eine Crew besteht meist aus einem oder mehreren ranghöheren Flugbegleiter*innen und den zugehörigen rangniederen Flugbegleiter*innen.
Die Dienstplanerstellung erfolgt ungefähr zwei Monate im Vorhinein und wird bis zu neun Tage vor Dienstantritt bekannt gegeben, 
dabei gibt es die Möglichkeit von Seiten der Arbeitgeber, auf konkrete Terminwünsche Rücksicht zu nehmen.
Die durchzuführenden Tätigkeiten der Crew sind hoch standardisiert. Es gibt klare Regeln wie und wann, was durchzuführen ist. 
Dazu haben die verschiedenen Teammitglieder genau definierte Rollen bezüglich ihrer Funktion.

Die Untersuchungsperson arbeitet, abhängig vom Dienstplan, sowohl auf Kurz- als auch auf Langstreckenflügen. 
Ihr Aufgabenbereich umfasst meist sicherheitsrelevante Tätigkeiten,
Einweisung von Passagieren, die Kontrolle des Handgepäcks und den Bordservice.
Trotz der hohen Standardisierung sind Arbeitszeiten, Aufenthaltsorte und Tagesabläufe oft sehr unterschiedlich.



\subsection{Beschreibung der Fragestellung}
Im Rahmen des vorliegenden Analyseprojekts setzen wir uns mit der Tätigkeit der Flugbegleiterin 
Frau Z. die seit X Jahren bei X Airline tätig ist. Dabei liegt der Fokus auf der folgenden 
Fragestellung:
Mit welchen Anforderungen ist Frau Z. in ihrer Tätigkeit konfrontiert und
welche Ressourcen stehen ihr zur Bewältigung zur Verfügung? Durch welche Maßnahmen kann der empfundene 
Stress reduziert und die Ressourcen zur Bewältigung erhöht werden? 

\section{Vorgehen}

Der Feldzugang zur Untersuchungsperson erfolgte über eine persönlichen Kontakt eines Mitglieds 
unserer Gruppe. Dabei wurde Frau Z. über as Vorhaben und den geplanten Ablauf informiert und 
es wurden erste Informationen bezüglich ihrer Tätigkeit erhoben. Basierend drauf und 
unseren Intressenschwerpunkten aufbauend haben wir unsere Fragestellung entwickelt. 

Anschließend wurden die Termine für den quantiativen Fragebogen und das quliative Interview 
mit Frau Z. vereinbart. Dabei wurden die Termine so gewählt, dass im Interview auf die Ergebnisse 
des Fragebogens eingegangen werden konnte. 

Nach sorgfältiger Analyse des Theoretischen 
hintergrunds und der Analysekriterien wurde der ISTA, als qualitatives instrument zur 
Stressbezogenen Tätigkeitsanalyse ausgewählt.

Der Fragebogen wurde der Untersuchungspseron zum vereinbartem Zeitpunkt digital übermittelt und 
von dieser nach der Bearbeitung ausgefüllt zurückgeschickt. Die Ergebnisse des Fragebogens wurden 
von der Gruppe auf Auffälligkeiten analysiert um draus einzelne Fragen für das Anschließende 
Interview abzuleiten. 

Daraufhin wurde ein Interview Leitfaden erstellt. 
Nach einholen des Einverständnisses der Versuchsperson wurde das Interview von zwei Mitgliedern 
der Gruppe an einem öffentlichem Ort durchgeführt und aufgezeichnet. Das Interview wurde
mithilfe einer Software transkribiert und von Teilen der Gruppe ausgewertet. 

Anschließend wurden die Ergebnisse des ISTA und des Interviews final interpretiert und 
geigente Maßnahmen abgleitet. 



\subsection{Quantitative Befragung: ISTA}

\subsubsection{Das Testverfahren}

Das von uns ausgewählte Testverfahren, um unsere Fragestellung zu untersuchen, 
ist der ISTA: Instrument zur Stressbezogenen Tätigkeitsanalyse 
\parencite{semmerISTAInstrumentZur1998}. 
Die theoretische Grundlage dieses Tests ist das Transaktionale Stressmodell 
welches sich mit dem Konstrukt aus unserer Fragestellung deckt und Stress
als Produkt von Diskrepanz zwischen wahrgenommenen Ressourcen und externen, 
sowie internen Anforderungen sieht (Lazarus und  Folkmann, 1984) 
und die Handlungsregulationstheorie (Hacker, 1986; Volpert,
1987). 
Der Einfluss der objektiven Faktoren wird von den Autoren des ISTA hierbei jedoch
stärker gewichtet als es in der originalen Theorie von Lazarus und Folkmann der Fall ist.
Das Ziel des Fragebogens ist es Merkmale der Arbeit zu erheben, die stressrelevant sind,
wozu sowohl Ressourcen als auch Stressoren erhoben werden mit dem Ziel einer
Abschätzung der Folgen für Maßnahmen zur Arbeitsgestaltung. Die Durchführung des
Fragebogens dauert etwa eine Stunde (Semmer et al.,1999).
Der Fragebogen liegt in zweifacher Ausfertigung vor, einmal als Ratingversion für
Untersucherinnen und einmal für Arbeitnehmerinnen, wobei in unserer Testung letztere
Version angewandt wird. Das Verfahren wurde entwickelt, um insbesondere
Belastungsschwerpunkte von Produktions- und Büroarbeitstätigkeiten zu untersuchen. Da
die Fragen innerhalb des Fragebogens aber nicht auf konkrete Tätigkeiten der Büroarbeit
ausgerichtet sind, liegt unseres Erachtens nach kein Widerspruch für unsere Anwendung im
Bereich der Anforderungserhebung für den Job als Flugbegleiter:in vor. Um sicherzugehen,
dass die Antworten auf die Fragen auch von uns richtig verstanden wurden, ist eine
Nachbesprechung im Interview mit der Testperson nach dem Ausfüllen des Fragebogens
angedacht und etwaige Auffälligkeiten einer möglichen Beeinflussung des
Analyseergebnisses aufgrund unserer Wahl des Fragebogens werden im Abschlussbericht
von uns herausgehoben

\subsubsection{Gütekriterien}

Vorab ist anzumerken, dass die aktuellste Version des ISTA, laut über 

seit über 25 Jahren nicht
erneuert wurde bzw. im Handbuch darauf hingewiesen wurde, dass sich das Verfahren zu
der Zeit in Erprobung befindet. Aus diesem Grund haben wir uns dazu entschieden, in der
Auswertung auch auf eine Meta-Analyse von \textcite{irmerInstrumentStressOrientedTask2019}
zurückzugreifen, um aktuellere Vergleichswerte heranziehen zu können. 
Zum aktuellen Zeitpunkt der Testung liegt laut unserer Recherche keine Normwerttabelle 
für unsere zukünftigen Auswertungen und Interpretation vor.

Zur Erfassung der Reliabilität wurden im Handbuch lediglich Analysen der internen
Konsistenz der Skalen durchgeführt, welche für den Großteil der Skalen ausreichende Werte
ergaben (Fragebogenversion: r = 0,58-0,88; Ratingversion: r = 0,51-0,93) 
\parencite{semmerInstrumentZurStressbezogenen1999}. 
Daten bezüglich der Re-Test Reliabilität oder anderen Reliabilitätsmessungen sind in
dem uns vorliegenden Handbuch nicht angegeben. 
Die Kriteriumsvalidität wird mit $r$= 0,2-0,3
angegeben, was laut den Herstellern ein üblicher Wert für Stressmessungen ist.
Die Objektivität ist aufgrund der fehlenden Normtabellen ebenfalls kritisch zu betrachten. 

Es besteht eine Anleitung für die Auswertung, jedoch ist ein gewisser Spielraum für
Interpretation dieser gegeben, so ist zum Beispiel nicht definiert, was genau als “auffälliger
Wert” gilt, was wiederum mit dem Fehlen der Normtabellen zusammenhängt.

In der von uns erwähnten Metaanalyse werden jedoch eine interne Konsistenz von 
$r = 0,71-0,84$ und eine Re-Testreliabilität von $r =0,75-0,85$ angegeben, mit Ausnahme der
einseitigen Belastung, die mit $r$= 0,54 angegeben wurde.

Bei der Erhebung der Konstruktvalidität korrelieren alle erfragten Stressoren, außer der
Dimension Unfallgefährdung, signifikant positiv mit emotionaler Erschöpfung und
psychosomatischen Beschwerden.

Das Verfahren wird von den Herstellern und der Metaanalyse als ausreichend valide und
reliabel, sowie ökonomisch eingeschätzt und erfüllt somit die Anforderungen der drei
Hauptgütekriterien für den Rahmen des Seminars im Allgemeinen ausreichend und ist dem
zeitlichen Aufwand der Arbeitnehmer*in für die Analyse entsprechend. Zudem basiert das
Testverfahren, wie bereits erwähnt, zu einem Teil auf dem transaktionalen Stressmodell und
entspricht somit dem theoretischen Konstrukt, welches auch unserer Fragestellung zu
Grunde liegt. Die vorliegenden Aspekte haben uns zu der Entscheidung geführt, diesen
Fragebogen für unsere Erhebung zu benutzen.

\subsection{Qualitative Befragung: Halbstrukturiertes Interview}

Der Interviewleitfaden wurde basierend auf den Ergebnissen der Datenerhebung durch den 
ISTA und den Inhalten der Fragestellung erstellt. Als Format wurde dabei ein 
Halbstrukturierter Leitfaden gewählt. Dieses ermöglicht gezieltes Erfragen interessierender
Punkte und bietet gleichtzeitg die Flexibilität,mit spontanen Nachfragen, eventuell aufkommende Themen
Schwerpunkte genauer zu untersuchen. 

\subsubsection{Leitfaden}

\begin{tcolorbox}[
  breakable,
  colback=gray!10,
  colframe=black,
  arc=10pt,          % abgerundete Ecken
  boxrule=0.5pt,
  left=6pt, right=6pt, top=6pt, bottom=6pt
]

\small

\textbf{1. Allgemeine Begrüßung vor Interview}\\

Danke, dass du dir heute Zeit genommen hast für dieses Interview. Wir werden dir heute
Fragen zu deinen Erfahrungen als Flugbegleiterin stellen und dabei insbesondere den Fokus
auf Belastungsfaktoren und Ressourcen zur Aufgabenbewältigung, die in deiner Arbeit
wichtig sind, setzen. Für unsere Auswertung würden wir dieses Interview aufzeichnen und
uns zusätzlich Notizen machen. Das ganze wird ungefähr eine Stunde dauern,
selbstverständlich werden deine Daten vertraulich behandelt und nicht an dritte Personen
oder Programme weitergegeben, außerdem hast du jederzeit die Möglichkeit einzelne
Fragen nicht zu beantworten oder das Interview ohne Angabe eines Grundes abzubrechen.
Bitte beantworte die Fragen so ausführlich wie möglich, es gibt keine falschen Antworten, da
es hier um deine persönlichen Erfahrungen geht. Wenn du keine weiteren Fragen zum
Ablauf hast und du mit allen genannten Punkten einverstanden bist, werden wir das
Interview jetzt starten.\\

\textbf{2. Allgemeine Fragen zur Job-Beschreibung}

\begin{itemize}
    \item Wie würdest du einer interessierten Person so genau wie möglich erklären, was du
    beruflich machst?
    \item Welche Hauptaufgaben umfassen dein Job als Flugbegleiterin?
    \item Beschreibe einmal ausführlich den Ablauf eines typischen Arbeitstages - vom Anfang
    bis zum Ende - (Weitere Nachfrageoptionen:)
    \begin{itemize}
        \item Wie stark variiert dein Tagesablauf?
        \item Wie sehen deine Arbeitszeiten normalerweise aus?
        \item Wie werden Aufgaben verteilt?
        \item Wie viel Freiheit hast du in der Umsetzung deiner Aufgaben?
        \item Wie erlebst du die Zusammenarbeit mit dem Team/Kollegen?
        \item Wie ist die Hierarchie strukturiert?
        \item Wie äußert sich die Zusammenarbeit mit deinen Vorgesetzten?
        \item Was für Aufstiegsmöglichkeiten gibt es?
        \item Wie erlebst du die Arbeit mit den Passagieren?
        \item Wie sieht die Pausengestaltung aus?
        \item Wie laufen berufliche Übernachtungen ab?
        \item Wie erlebst du den räumlichen Arbeitskontext als Flugbegleiterin im
        Flugzeug?
        \item Welche Rückzugsmöglichkeiten hast du?
        \item Inwiefern beeinflussen unterschiedliche Zeitzonen deinen Arbeitsalltag?
        \item Seit wann arbeitest du in dem Unternehmen?
        \item Hat sich etwas über die Zeit verändert?
    \end{itemize}
\end{itemize}

\textbf{3. Nachfragen zum Fragebogen} \\

Danke für deine Antworten. Das gibt uns schon mal einen sehr guten Überblick über deine
Arbeit. Im nächsten Teil des Interviews wollen wir auf deine Erfahrungen mit dem
Fragebogen eingehen.
\begin{itemize}
    \item Wie ist es dir bei der Beantwortung der Fragen ergangen?
    \item Welche Aspekte sind dir in Erinnerung geblieben, die du als besonders relevant
    empfindest?
    \item Welche Themen haben dir bei der Beantwortung gefehlt?
    \item $[$Platzhalter für Fragen zu Auffälligkeiten$]$ 
\end{itemize}

\textbf{3. Herausfordernde Ereignisse} \\

Danke für die Rückmeldungen bezüglich des Fragebogens, das ist sehr wichtig für uns, wir
werden das in unserem Abschlussbericht berücksichtigen. Gibt es noch etwas, dass du zu
dem vorher besprochenen ergänzen wollen würdest? Okay, dann würden wir uns jetzt eine
konkrete Situation in deiner Arbeit anschauen.

\begin{itemize}
    \item  Hierzu würden wir dich bitten, uns durch ein für deine Arbeit charakteristisches
    Szenario zu führen, das du als herausfordernd erlebt hast. Beginne hierfür in deiner
    Darstellung bitte schon etwas vor dem Beginn der Situation und beschreibe es uns
    dabei so, dass wir es uns wie einen Film vorstellen können.
    \begin{itemize}
        \item Wie hast du dich dabei gefühlt?
        \item Welche Faktoren stellen in dieser Situation konkret die Herausforderung für
        dich dar?
        \item Was hätte es gebraucht, damit diese Situation besser bewältigt hätte werden
        können?
    \end{itemize}
    \item Wie häufig passieren solche oder ähnlich herausfordernde Situationen?
    \begin{itemize}
        \item Welche Rolle spielt Teamarbeit in solchen Situationen?
        \item Wie unterstützt dich dein Arbeitgeber bei der Ausführung herausfordernder
        Tätigkeiten?
        \item Wie wirken sich Pausen auf dein Erleben oder deine Bewältigung solcher
        Situationen aus?
        \item Welchen Unterschied macht es für dich persönlich, wenn du dich auf einem
        Langstreckenflug- oder Kurzstreckenflug befindest?
        \item Welchen Entscheidungs- und Handlungsspielraum hast du hierbei?
        \item Wie viel Einfluss hast du auf die Zielgebiete, zu denen du fliegst?
        \item Wie wirkt sich deine Tätigkeit auf dein Privatleben aus?
        \item Wie geht dein Arbeitgeber mit eventuellen Krankenständen um?
        \item Wie erlebst du Passagiere in so einer Situation?
        \item Wie sieht es mit Sprachbarrieren aus?
        \item Wie gehst du mit unvorhergesehenen bzw. ungeplanten Situationen um?
        \item Wie fühlst du dich in solchen Situationen?
    \end{itemize}
\end{itemize}

\textbf{4. Abschluss} \\

Vielen Dank, dass du deine Erfahrungen mit uns geteilt hast.
Gibt es noch etwas, das du gerne erwähnen würdest, 
worüber wir bis jetzt noch nicht geredet haben?

\end{tcolorbox}

\section{Ergebnisse}

\subsection{Quantitative Ergebnisse}

\begin{table}[h]
    \centering
    \begin{threeparttable}
        \caption{ISTA Inhalt \parencite{semmerInstrumentZurStressbezogenen1999}}
        \label{ITSA_Dims}
        \small
        \begin{tabularx}{\dimexpr\textwidth}{lcX}
            \toprule
            Skala/Index & Item Anzahl & Beschreibung \\
            \midrule
            %Soziodemographische Angaben & 8 & Welche Berufsausbildung haben Sie in Bezug auf Ihre jetzige Tätigkeit? \\
            %Qualifikationserfordernisse & 3 & Wenn Sie ihre Tätigkeit insgesamt betrachten wie viel Qualifikation verlangt sie? \\
            Komplexität & 5 & Komplexität der Anforderungen  \\
            Handlungsspielraum & 5 & Entscheidungsmöglichkeiten über Vorgehensweise
            und Reihenfolge \\
            Partizipation & 7 & Einfluss auf Urlaubspläne, Arbeitszeit, Anschaffungen
            u. ä.\\
            Zeitspielraum & 5 & Einfluss auf Zeiteinteilung \\
            Variabilität & 5 & Verschiedenartigkeit der Anforderungen \\
            Unsicherheit & 5 & Unsicherheit über Anforderungen, Arbeitsergebnisse,
            Folgen \\
            Unfallgefährdung & 5 & --  \\
            Arbeitsorganisatorische Probleme & 5 & Qualität von Unterlagen, Material,
            Einrichtung des Arbeitsplatzes, Zwickmühle Qualität / Quantität \\
            Einseitige Belastung & 6 & Langes Stehen/Sitzen, gebeugte, verdrehte u.ä.
            Körperhaltung \\ 
            Umgebungsbelastungen & 17 & Ausprägung von Lärm, ungünstiger Beleuchtung 
            usw. \\
            Arbeitsunterbrechungen & 5 & Unterbrechungen durch Vorgesetzte, Kollegen,
            Eilaufträge u. ä. \\
            Konzentrationsanforderungen & 5 & Vigilanzanforderungen, Belastung des 
            Kurzzeitgedächtnisses \\
            Zeitdruck & 5 & Hohes Arbeitstempo / -volumen \\
            Kommunikationsmöglichkeiten & 3 & Möglichkeiten der Kontaktaufnahme, incl.
            nicht arbeitsbezogener Kommunikation \\
            Kooperationsspielraum & 3 & Möglichkeit zur gegenseitigen Unterstürzung,
            Einfluss auf Auswahl von Koop.-PartnerInnen \\
            Kooperationsenge & 5 & Abhängigkeit von anderen (negative Aspekte von
            Kooperation) \\
            Kooperationserfordernisse & 4 & Notwendigkeit gemeinsamer Entscheidungen
            und gegenseitiger Informationen \\ 
            \bottomrule
        \end{tabularx}
        \begin{tablenotes}[flushleft]
            \footnotesize
            \item Notes: Test
        \end{tablenotes}
    \end{threeparttable}

\end{table}


\begin{table}[h!]
    \centering
    \begin{threeparttable}
    \caption{ISTA Scores}
    \begin{tabularx}{\dimexpr\textwidth}{lrrX}
        \toprule
        Dimension & z-Wert & PR & Beschreibung\\
        \midrule
        Komplexität & $-2.94$ & 0 & Komplexität ist deutlich unterdurchschntlich ausgeprägt, was darauf hinweißt, dass Frau Z.
        kaum mit komplexen Entscheidung konfrontiert ist und sehr wenig herausfordernde Aufträge erhält \\
        Handlung Spielraum & -2.17 & 2 & ---\\
        Partizipation & -1.51 & 7 & ---\\
        Variabilität & -0.52  & 30 & --- \\
        Zeitspielraum & -2.41 & 1 & ---\\
        Unsicherheit & 0.58 & 72 & ---\\
        Unfallgefährdung & 0.51 & 69 & ---\\
        Arbeitsorganisatorische Probleme & -0.12 & 45 & ---\\
        Einseitige Belastung & -0.90 & 18 &----\\
        Umgebungsbelastung & 1.41 & 92 &----\\
        Arbeitsunterbrechungen & 1.39 & 92 &--- \\
        Konzentrationsanforderungen & 0.85 & 80 &---\\
        Zeitdruck & 0.36 & 64 & ---\\
        Kommunikationsmöglichkeiten & 0.96 & 83  & ---\\
        Kooperationsspielraum & $1.67^a$ & -- & ---\\
        Kooperationsenge & $5^a$ & -- & --\\
        Kooperationserfordernisse & 0.91 & 82 & ---\\
        \bottomrule
    \end{tabularx}
    \end{threeparttable}
\end{table}


\begin{table}[h]
    \centering
    \begin{threeparttable}
    \caption{ISTA Scores}
    \begin{tabular}{lrrrrrr}
        \toprule
        Dimension & {$z$-Wert} & \multicolumn{2}{c}{95\% CI}  & PR & \multicolumn{2}{c}{95\% CI} \\
        \cmidrule(lr){3-4} \cmidrule(lr){6-7}
         & & \multicolumn{1}{c}{UG}  & \multicolumn{1}{c}{OG} & & \multicolumn{1}{c}{UG} & \multicolumn{1}{c}{OG} \\
        \midrule
        Arbeitsorganisatorische Probleme & $-0.12$ & $-0.34$ & $-0.23$ & 45 & 34 & 12\\
        Einseitige Belastung & -0.90 & 0.23 & 0.34 & 18 & 34 & 12 \\
        Einseitige Belastung & 0.90 & 0.23 & 0.34 & 18 & 34 & 12 \\
        \bottomrule
    \end{tabular}
    \begin{tablenotes}[flushleft]
        \small
        \item \textit{Anmerkung:} PR = Prozentrang
    \end{tablenotes}
    \end{threeparttable}
\end{table}

\subsection{Qualitative Ergebnisse}

Informationsfluss

Die Interviewperson berichtet von kaum Informationsfluss zwischen vorgesetzten Personen und dem tatsächlich im Dienst anwesenden Team. 
In einem Ausschnitt des Interviews berichtet die Frau Z., unvorhergesehene Situationen wie z.B. Wetterumschwüngen würden gut kommuniziert werden.
Im nächsten Ausschnitt berichtet sie, über solche Situationen würde sie informiert werden, wenn die zuständigen Personen Zeit dafür hätten.
Laut unserer Interviewperson würden Dienstpläne am 21. des Monats für den darauffolgenden Monat bereitgestellt werden.

Vereinbarkeit Privat und Beruf

Außerhalb der tatsächlichen Arbeitszeit sei die Interviewperson dazu verpflichtet, sich auf einem Dienst-iPad über mögliche Dienständerungen
oder Anforderungen zu informieren. Zusätzlich seien in der Freizeit auch Vorbereitungen, wie sich um saubere Dienstkleidung zu kümmern, zu tätigen.
Abgesehen von diesen Tätigkeiten seien Beruf und Privatleben klar getrennt.
Die Interviewperson bezeichnet ihre Situation hinsichtlich Vereinbarkeit von Beruf 
und Privatleben an einer Stelle im Interview als “sehr einschränkend”, aufgrund von Diensten an Wochenend- und Feiertagen 
und dem Zeitpunkt der Dienstplan-Veröffentlichung.

Arbeitszeiten

Die Interviewperson berichtet von sehr unregelmäßigen Arbeitszeiten, welche in den Tageszeiten und in der Dienstlänge variieren würden.
Die Tätigkeit würde zudem geringer vergütet werden, wenn die Flugzeugtüren noch geöffnet seien und sich das Flugzeug noch am Boden befinden würde. 

Zeitdruck

Aufgrund der zuvor beschriebenen Arbeitszeitregelung steht die Interviewperson in einigen Situationen in ihrer Tätigkeit unter starkem Zeitdruck,
was sich auf das Arbeitsklima auswirken würde. 
Bei Kurzstreckenflügen sei ein größerer Zeitdruck vorhanden als bei Langstreckenflügen.
Dieser Zeitdruck steht stark in Zusammenhang mit den Möglichkeiten zur Pausengestaltung.

Ruhephasen und Pausen

Aufgrund der örtlichen Gegebenheiten gäbe es keinen geeigneten Rückzugsort, in einigen Fällen könne hierfür die Küche benutzt werden,
diese sei aber nicht in jedem Flugzeug vorhanden.
Gerade bei Kurzstreckenflügen sei die Möglichkeit, Pausen zu machen, sehr situationsabhängig, 
je nach Passagieranfragen, Kollegenanfragen und anfallenden Aufgaben. 
Es gäbe hierbei keine festgelegten Pausen. Bei Langstreckenflügen sei die Interviewperson etwas freier in ihrer Pausengestaltung.
Bei bestimmten Flügen gäbe es festgelegte, verpflichtende Zeiten, um zu schlafen. 

Hierarchie

Es wurde eine strenge Hierarchie beschrieben: Im gesamten Unternehmen wäre die Hierarchie klar geregelt.
Innerhalb einzelner Dienste könne die Position der Interviewperson variieren, je nachdem, wie lange sie im Vergleich zu anderen
Teammitgliedern dabei wäre. Laut unserer Interviewperson dürfe sie als höhergestellte Person beliebtere Aufgaben übernehmen und müsse als 
niedrig gestellte Person unbeliebtere Aufgaben machen. Da sich die Zusammensetzung des Teams je nach Dienst unterscheide, 
würde unsere Interviewperson unterschiedliche Positionen einnehmen. Die höchste Position für einen Dienst sei jedoch im Vorhinein festgelegt. 

Variabilität im Tätigkeitsablauf

Laut der Interviewperson würden die Aufgaben in ihrer Tätigkeit wenig variieren und es gäbe strikte Anweisungen für anfallende Szenarien.
Ihre Tätigkeit würde jedoch darin variieren, ob sie Dienste über mehrere Tage hinweg hätte oder ob es sich um kürzere Dienste handeln würde. 

Beeinflussbarkeit der Umgebung

Da viele der im Interview beschriebenen Szenarien eine systembedingte Ursache hätten, könne man nicht viel an den Gegebenheiten des Jobs ändern.
Die Beeinflussbarkeit des Settings, sowie auch der aufkommenden Herausforderungen wurde somit als sehr niedrig beschrieben. 
Allerdings hätte man die Möglichkeit auch Wünsche für den eigenen Dienstplan anzugeben, die aber nicht immer berücksichtigt würden.

Arbeitsfluss

Die Interviewperson berichtet von einem strengen Arbeitsablauf, der genau zu befolgen wäre.
Auch könne man erst in die Pause gehen, wenn alle Arbeiten erledigt wären. 
Manchmal käme es zu Unterbrechungen der aktuellen Arbeitstätigkeit aufgrund von dringenden Aufgaben, die akut erledigt werden müssten.

Team

Das Team hätte aufgrund der von Flug zu Flug wechselnden Arbeitskolleg*innen sowohl eine unterstützende als auch herausfordernde Komponente, 
besonders deshalb, da man vor einem Dienst nie wisse, mit wem man zusammen arbeite.
Es könne z.B. immer wieder zu Konflikten innerhalb der Teams kommen, aber es gäbe auch gegenseitige 
Unterstützung in der Arbeit mit den Passagieren und es könnten teilweise sogar Freundschaften mit einigen geschlossen werden, die über ein reines Arbeitsverhältnis hinausgingen.

Emotionale Anforderungen

Zu den emotionalen Belastungen wären die Konflikte innerhalb des Teams und mit Passagieren zu zählen.
Bei letzteren müsse man immer freundlich bleiben und es herrsche wenig Ausweichraum, vor allem wenn man sich gerade im Flugzeug befände.
Zudem würde eben jener enge Raum des Arbeitsumfeldes als belastend empfunden werden, sowie die wenigen Möglichkeiten der Pausengestaltung. 
Auch die Luft innerhalb des Flugzeuges und der durch die Motoren verursachte Lärm wurden als Negativbeispiele genannt. 
Als letzten Punkt erwähnte die Interviewperson die Möglichkeit, in Gebiete geschickt werden zu können, 
in denen gerade Unruhen herrschten, ohne adäquate Möglichkeit dies abzulehnen.

Körperliche Anforderungen

Aufgrund der arbeitsbedingten unterschiedlichen Zeitzonen und Dienstzeiten entstünden unregelmäßige und verkürzte Schlafzeiten,
welche als belastend für den Körper erlebt würden.

Arbeitsorganisatorisches

Die Interviewperson berichtet, dass man sich um das Mitnehmen der Uniform, sowie auch der Wechselkleidung für mehrtägige
Dienste selbst kümmere und dies nicht vom Arbeitgeber zur Verfügung gestellt werde.
Andererseits würden Hotels mit spezifischer verpflichtender Ausstattung (Black-Out Vorhänge etc.) vom Arbeitgeber organisiert werden.
Außerdem müsse ein Taxi vom Flughafen direkt zum Hotel, wie auch zurück zum Flughafen, bereitgestellt werden.

Mitarbeiter Benefits

Es gäbe jährliche Schulungen zu Konfliktmanagement, die verpflichtend besucht werden müssten.
Als positiv wurde zudem hervorgehoben, dass der Job die Möglichkeit biete, viele verschiedene Orte sehen zu können und
auch außerhalb der Saisonzeiten zu vergünstigten Preisen Flüge zu buchen.

\newpage
\printbibliography

\end{document}
